%% LyX 2.2.3 created this file.  For more info, see http://www.lyx.org/.
%% Do not edit unless you really know what you are doing.
\documentclass{article}
\usepackage{fontspec}
\setmainfont[Mapping=tex-text]{Source Han Serif SC}
\setsansfont[Mapping=tex-text]{Source Han Sans SC}
\usepackage{geometry}
\geometry{verbose,tmargin=2.5cm,bmargin=2.5cm,lmargin=2.5cm,rmargin=2.5cm}
\usepackage{color}
\usepackage[unicode=true,pdfusetitle,
 bookmarks=true,bookmarksnumbered=false,bookmarksopen=false,
 breaklinks=false,pdfborder={0 0 0},pdfborderstyle={},backref=false,colorlinks=true]
 {hyperref}

\makeatletter
%%%%%%%%%%%%%%%%%%%%%%%%%%%%%% User specified LaTeX commands.
\usepackage[linguistics]{forest}
\usepackage[all]{xy}
\usepackage{minted}
\usepackage{indentfirst}
\setlength{\parindent}{2em}
\XeTeXlinebreaklocale "zh"
\XeTeXlinebreakskip = 0pt plus 1pt minus 0.1pt
\setminted{
baselinestretch=1.2,
fontsize=\footnotesize,
linenos}

\makeatother

\usepackage{xunicode}
\begin{document}

\title{编译原理实验一:词法分析与语法分析}

\author{郭松 2015301500205}
\maketitle

\section{系统环境}
\begin{verbatim}
Ubuntu 17.10 (Kernel 4.13.0-16) GCC 7.2.0 Flex 2.6.1 Bison (GNU Bison 3.0.4) 
\end{verbatim}

\section{功能简介}

对于有一定错误的程序,也尝试生成抽象语法树。基本完成了必做样例和选做样例的全部内容。

构建抽象语法树,并给出每一个节点对应的源代码(的范围)。

在附件中准备了很多.c文件,是我自己用来测试功能的,涵盖了大部分语法点。

另外,Makefile文件是非常简单的Make脚本,直接调用make即可编译。如果使用make debug编译,可以生成含调试输出的程序,能输出更详细的错误信息。

当程序遇到了词法错误,会产生类似于下面的输出:

\begin{minted}{text}
line 4: <<Error Type A.2>> Nya! I cannot recognize ``~'', wtf!
line 5: <<Error Type A.1>> Nya! I can recognize ``1e'', but why?
line 6: <<Error Type A.0>> Nya? ``0x1G'' might be a wrong hex integer.
line 7: <<Error Type A.0>> Nya? ``09'' might be a wrong oct integer.
\end{minted}

当程序遇到了语法错误,会产生类似于下面的输出:

\begin{minted}{text}
line 8: syntax error
line 8: <<Error Type B.1>> Meow! Valid argument expression required.
line 9: syntax error
line 9: <<Error Type B.1>> Meow! Valid expression required.
line 10: syntax error
line 10: <<Error Type B.0>> Meow? ``;'' is expected
\end{minted}

和要求的输出格式有一些不同,我的输出格式为:

\begin{minted}{text}
[  1:  1]->[ 21:  1]Program 
[  1:  1]->[ 21:  1]  ExtDefList 
[  1:  1]->[  4:  1]    ExtDef 
[  1:  1]->[  1:  5]      Specifier 
[  1:  1]->[  1:  5]        TYPE: float 
[  1:  7]->[  1: 20]      FunDec 
[  1:  7]->[  1: 11]        ID: sqr_f 
[  1: 12]->[  1: 12]        ( 
[  1: 13]->[  1: 19]        VarList 
[  1: 13]->[  1: 19]          ParamDec 
[  1: 13]->[  1: 17]            Specifier
(...)
\end{minted}

每一行的开头表示了语法树节点对应的源代码中的位置。同时对于部分终结符也只显示其文本(如各种括号,一些运算符等)

\section{功能测试}

调用make编译完成之后,会产生一个叫”ejq\_cc”的程序,即最终的可执行文件。

我在附件的test文件夹中准备了诸多用于测试的样例。

ac.c是一个简单的程序,包含了基本的语法元素

ac\_numbers.c是一个仅包含数值字面量的程序,展示了对于数字的词法分析

normal.c和dinic.c是两个由实际环境下的程序修改得到的程序,展示了在一般情况下的表现基本覆盖了所有语言点。

error{*}.c是一些简短的,包含了常见错误的C语言程序。

\section{词法分析}

\subsection{十进制整数}

十进制整数不含有前导零,即如果这个数非零,那么它首位不为0,否则其为0,根据这个定义,可以写出:

\begin{minted}{perl}
([1-9][0-9]*)|0
\end{minted}

其前半部分表示正数,后半部分表示0,对于负数的表示,我们将其表示为符号后跟一个正数,这也是大部分C编译器的实现。

\subsection{八进制整数}

八进制整数以0开头,不包含有大于7的数码,可以有前导零,因而可以写为

\begin{minted}{perl}
OCTINT 0[0-7]+
ERROCT 0[0-9]+ 
\end{minted}

这里,Erroct表示了猜测为错误的八进制整数的情况。

\subsection{十六进制整数}

十六进制整数以0x开头,包含0-9和a-f,不区分大小写,因而可以写为

\begin{minted}{perl}
HEXINT 0[xX][0-9a-fA-F]+
ERRHEX 0[xX][0-9a-zA-Z_]+
\end{minted}

这里Errhex表示了猜测为错误的十六进制整数的情况。

\subsection{十进制浮点数}

十进制浮点数有两种表示方法,其一为普通的表示的方法,其二为科学计数法。对于普通的表示方法,小数点前后至少一部分不为空,则可以分情况考虑为

\begin{minted}{perl}
{INT}.[0-9]* | .[0-9]+
\end{minted}

对于第二种表示方法,可以认为其是简单地在后加上了指数部分,那么指数部分可以表示为

\begin{minted}{perl}
[eE][+-]?{INT}
\end{minted}

同时也要注意到前半部分也可以是一个整数。因此综合上述两种情况可以表示为:

\begin{minted}{perl}
(({INT}(\.{DIGIT}*)?|\.{DIGIT}+)([eE][+-]?{DIGIT}+)|({INT}?\.{DIGIT}+)|({INT}\.{DIGIT}*))
\end{minted}

\subsection{行末注释的表示}

行末注释可以表示为''//''.{*}''\textbackslash{}n'',Flex使用的正则表达式的''.''不包含换行符,十分方便。

\subsection{块注释的表示}

因为注释不组成任何一个语法符号,所以考虑使用词法分析完成块注释的隔离。Flex提供了“状态”这一概念,可以给自动机加入指定的状态,因此,在读入''/{*}''之后,我们进入comment状态,直到读到第一个''{*}/''为止,返回INITAL状态。这样的处理方式天然解决了嵌套注释的问题。

对于其余的正则表达式,十分简单,不再赘述。

\subsection{错误处理}

按照词法定义,不包含在词法定义中的单词都被认为是错误的单词,在正则表达式中用

\begin{minted}{c}
ERRWORD ([a-zA-Z0-9_]+)
%%
{ERRWORD} {
    char buf[1024];
    sprintf(buf, "<<Error Type A.1>> Nya! I can recognize ``%s'', but why?", yytext);
    yyerror(buf);
    sprintf(buf, "ERRWORD: %s", yytext);
    yylval = nnewnode(buf, lineno, charno, lineno, charno + strlen(yytext));
    charno += strlen(yytext);
    return ID;
}
\end{minted}

来表示。因为yyerror的签名不能格式化,所以需要用sprintf来格式化错误信息。在这一种情况下,我将其认定为一个ID

\section{文法分析}

对于基本的文法分析,照着要求的附件翻译一下即可。对于错误处理,也只需稍加修改,在可能的地方加上error标记,然后处理即可,这里只介绍构建语法树的方法。

定义语法树结构体:

\begin{minted}{c}
typedef struct node{
    char *desc;
    int soncnt;
    int start_lineno, start_pos, end_lineno, end_pos;
    struct node** son;
}node;
\end{minted}

desc起到了描述的功能,在日后进行修改的时候,如要添加访问标志符表等需求,只需稍加修改如添加item项,即可实现。 start\_lineno、start\_pos、end\_lineno和end\_pos是为了描述语法树节点对应的源代码的位置。
\end{document}
